
%% bare_conf.tex
%% V1.3
%% 2007/01/11
%% by Michael Shell
%% See:
%% http://www.michaelshell.org/
%% for current contact information.
%%
%% This is a skeleton file demonstrating the use of IEEEtran.cls
%% (requires IEEEtran.cls version 1.7 or later) with an IEEE conference paper.
%%
%% Support sites:
%% http://www.michaelshell.org/tex/ieeetran/
%% http://www.ctan.org/tex-archive/macros/latex/contrib/IEEEtran/
%% and
%% http://www.ieee.org/

%%*************************************************************************
%% Legal Notice:
%% This code is offered as-is without any warranty either expressed or
%% implied; without even the implied warranty of MERCHANTABILITY or
%% FITNESS FOR A PARTICULAR PURPOSE! 
%% User assumes all risk.
%% In no event shall IEEE or any contributor to this code be liable for
%% any damages or losses, including, but not limited to, incidental,
%% consequential, or any other damages, resulting from the use or misuse
%% of any information contained here.
%%
%% All comments are the opinions of their respective authors and are not
%% necessarily endorsed by the IEEE.
%%
%% This work is distributed under the LaTeX Project Public License (LPPL)
%% ( http://www.latex-project.org/ ) version 1.3, and may be freely used,
%% distributed and modified. A copy of the LPPL, version 1.3, is included
%% in the base LaTeX documentation of all distributions of LaTeX released
%% 2003/12/01 or later.
%% Retain all contribution notices and credits.
%% ** Modified files should be clearly indicated as such, including  **
%% ** renaming them and changing author support contact information. **
%%
%% File list of work: IEEEtran.cls, IEEEtran_HOWTO.pdf, bare_adv.tex,
%%                    bare_conf.tex, bare_jrnl.tex, bare_jrnl_compsoc.tex
%%*************************************************************************

% *** Authors should verify (and, if needed, correct) their LaTeX system  ***
% *** with the testflow diagnostic prior to trusting their LaTeX platform ***
% *** with production work. IEEE's font choices can trigger bugs that do  ***
% *** not appear when using other class files.                            ***
% The testflow support page is at:
% http://www.michaelshell.org/tex/testflow/



% Note that the a4paper option is mainly intended so that authors in
% countries using A4 can easily print to A4 and see how their papers will
% look in print - the typesetting of the document will not typically be
% affected with changes in paper size (but the bottom and side margins will).
% Use the testflow package mentioned above to verify correct handling of
% both paper sizes by the user's LaTeX system.
%
% Also note that the "draftcls" or "draftclsnofoot", not "draft", option
% should be used if it is desired that the figures are to be displayed in
% draft mode.
%
\documentclass[conference]{IEEEtran}
% Add the compsoc option for Computer Society conferences.
%
% If IEEEtran.cls has not been installed into the LaTeX system files,
% manually specify the path to it like:
% \documentclass[conference]{../sty/IEEEtran}





% Some very useful LaTeX packages include:
% (uncomment the ones you want to load)


% *** MISC UTILITY PACKAGES ***
%
%\usepackage{ifpdf}
% Heiko Oberdiek's ifpdf.sty is very useful if you need conditional
% compilation based on whether the output is pdf or dvi.
% usage:
% \ifpdf
%   % pdf code
% \else
%   % dvi code
% \fi
% The latest version of ifpdf.sty can be obtained from:
% http://www.ctan.org/tex-archive/macros/latex/contrib/oberdiek/
% Also, note that IEEEtran.cls V1.7 and later provides a builtin
% \ifCLASSINFOpdf conditional that works the same way.
% When switching from latex to pdflatex and vice-versa, the compiler may
% have to be run twice to clear warning/error messages.






% *** CITATION PACKAGES ***
%
%\usepackage{cite}
% cite.sty was written by Donald Arseneau
% V1.6 and later of IEEEtran pre-defines the format of the cite.sty package
% \cite{} output to follow that of IEEE. Loading the cite package will
% result in citation numbers being automatically sorted and properly
% "compressed/ranged". e.g., [1], [9], [2], [7], [5], [6] without using
% cite.sty will become [1], [2], [5]--[7], [9] using cite.sty. cite.sty's
% \cite will automatically add leading space, if needed. Use cite.sty's
% noadjust option (cite.sty V3.8 and later) if you want to turn this off.
% cite.sty is already installed on most LaTeX systems. Be sure and use
% version 4.0 (2003-05-27) and later if using hyperref.sty. cite.sty does
% not currently provide for hyperlinked citations.
% The latest version can be obtained at:
% http://www.ctan.org/tex-archive/macros/latex/contrib/cite/
% The documentation is contained in the cite.sty file itself.






% *** GRAPHICS RELATED PACKAGES ***
%
\ifCLASSINFOpdf
  % \usepackage[pdftex]{graphicx}
  % declare the path(s) where your graphic files are
  % \graphicspath{{../pdf/}{../jpeg/}}
  % and their extensions so you won't have to specify these with
  % every instance of \includegraphics
  % \DeclareGraphicsExtensions{.pdf,.jpeg,.png}
\else
  % or other class option (dvipsone, dvipdf, if not using dvips). graphicx
  % will default to the driver specified in the system graphics.cfg if no
  % driver is specified.
  % \usepackage[dvips]{graphicx}
  % declare the path(s) where your graphic files are
  % \graphicspath{{../eps/}}
  % and their extensions so you won't have to specify these with
  % every instance of \includegraphics
  % \DeclareGraphicsExtensions{.eps}
\fi
% graphicx was written by David Carlisle and Sebastian Rahtz. It is
% required if you want graphics, photos, etc. graphicx.sty is already
% installed on most LaTeX systems. The latest version and documentation can
% be obtained at: 
% http://www.ctan.org/tex-archive/macros/latex/required/graphics/
% Another good source of documentation is "Using Imported Graphics in
% LaTeX2e" by Keith Reckdahl which can be found as epslatex.ps or
% epslatex.pdf at: http://www.ctan.org/tex-archive/info/
%
% latex, and pdflatex in dvi mode, support graphics in encapsulated
% postscript (.eps) format. pdflatex in pdf mode supports graphics
% in .pdf, .jpeg, .png and .mps (metapost) formats. Users should ensure
% that all non-photo figures use a vector format (.eps, .pdf, .mps) and
% not a bitmapped formats (.jpeg, .png). IEEE frowns on bitmapped formats
% which can result in "jaggedy"/blurry rendering of lines and letters as
% well as large increases in file sizes.
%
% You can find documentation about the pdfTeX application at:
% http://www.tug.org/applications/pdftex
\usepackage[utf8]{inputenc}
\usepackage[T1]{fontenc}
\usepackage[french, english]{babel}

\usepackage{graphicx}
\usepackage{caption}
\usepackage{amsmath,amsfonts,amssymb}
\usepackage{subcaption}




% *** MATH PACKAGES ***
%
%\usepackage[cmex10]{amsmath}
% A popular package from the American Mathematical Society that provides
% many useful and powerful commands for dealing with mathematics. If using
% it, be sure to load this package with the cmex10 option to ensure that
% only type 1 fonts will utilized at all point sizes. Without this option,
% it is possible that some math symbols, particularly those within
% footnotes, will be rendered in bitmap form which will result in a
% document that can not be IEEE Xplore compliant!
%
% Also, note that the amsmath package sets \interdisplaylinepenalty to 10000
% thus preventing page breaks from occurring within multiline equations. Use:
%\interdisplaylinepenalty=2500
% after loading amsmath to restore such page breaks as IEEEtran.cls normally
% does. amsmath.sty is already installed on most LaTeX systems. The latest
% version and documentation can be obtained at:
% http://www.ctan.org/tex-archive/macros/latex/required/amslatex/math/





% *** SPECIALIZED LIST PACKAGES ***
%
%\usepackage{algorithmic}
% algorithmic.sty was written by Peter Williams and Rogerio Brito.
% This package provides an algorithmic environment fo describing algorithms.
% You can use the algorithmic environment in-text or within a figure
% environment to provide for a floating algorithm. Do NOT use the algorithm
% floating environment provided by algorithm.sty (by the same authors) or
% algorithm2e.sty (by Christophe Fiorio) as IEEE does not use dedicated
% algorithm float types and packages that provide these will not provide
% correct IEEE style captions. The latest version and documentation of
% algorithmic.sty can be obtained at:
% http://www.ctan.org/tex-archive/macros/latex/contrib/algorithms/
% There is also a support site at:
% http://algorithms.berlios.de/index.html
% Also of interest may be the (relatively newer and more customizable)
% algorithmicx.sty package by Szasz Janos:
% http://www.ctan.org/tex-archive/macros/latex/contrib/algorithmicx/




% *** ALIGNMENT PACKAGES ***
%
%\usepackage{array}
% Frank Mittelbach's and David Carlisle's array.sty patches and improves
% the standard LaTeX2e array and tabular environments to provide better
% appearance and additional user controls. As the default LaTeX2e table
% generation code is lacking to the point of almost being broken with
% respect to the quality of the end results, all users are strongly
% advised to use an enhanced (at the very least that provided by array.sty)
% set of table tools. array.sty is already installed on most systems. The
% latest version and documentation can be obtained at:
% http://www.ctan.org/tex-archive/macros/latex/required/tools/


%\usepackage{mdwmath}
%\usepackage{mdwtab}
% Also highly recommended is Mark Wooding's extremely powerful MDW tools,
% especially mdwmath.sty and mdwtab.sty which are used to format equations
% and tables, respectively. The MDWtools set is already installed on most
% LaTeX systems. The lastest version and documentation is available at:
% http://www.ctan.org/tex-archive/macros/latex/contrib/mdwtools/


% IEEEtran contains the IEEEeqnarray family of commands that can be used to
% generate multiline equations as well as matrices, tables, etc., of high
% quality.


%\usepackage{eqparbox}
% Also of notable interest is Scott Pakin's eqparbox package for creating
% (automatically sized) equal width boxes - aka "natural width parboxes".
% Available at:
% http://www.ctan.org/tex-archive/macros/latex/contrib/eqparbox/





% *** SUBFIGURE PACKAGES ***
%\usepackage[tight,footnotesize]{subfigure}
% subfigure.sty was written by Steven Douglas Cochran. This package makes it
% easy to put subfigures in your figures. e.g., "Figure 1a and 1b". For IEEE
% work, it is a good idea to load it with the tight package option to reduce
% the amount of white space around the subfigures. subfigure.sty is already
% installed on most LaTeX systems. The latest version and documentation can
% be obtained at:
% http://www.ctan.org/tex-archive/obsolete/macros/latex/contrib/subfigure/
% subfigure.sty has been superceeded by subfig.sty.



%\usepackage[caption=false]{caption}
%\usepackage[font=footnotesize]{subfig}
% subfig.sty, also written by Steven Douglas Cochran, is the modern
% replacement for subfigure.sty. However, subfig.sty requires and
% automatically loads Axel Sommerfeldt's caption.sty which will override
% IEEEtran.cls handling of captions and this will result in nonIEEE style
% figure/table captions. To prevent this problem, be sure and preload
% caption.sty with its "caption=false" package option. This is will preserve
% IEEEtran.cls handing of captions. Version 1.3 (2005/06/28) and later 
% (recommended due to many improvements over 1.2) of subfig.sty supports
% the caption=false option directly:
%\usepackage[caption=false,font=footnotesize]{subfig}
%
% The latest version and documentation can be obtained at:
% http://www.ctan.org/tex-archive/macros/latex/contrib/subfig/
% The latest version and documentation of caption.sty can be obtained at:
% http://www.ctan.org/tex-archive/macros/latex/contrib/caption/




% *** FLOAT PACKAGES ***
%
%\usepackage{fixltx2e}
% fixltx2e, the successor to the earlier fix2col.sty, was written by
% Frank Mittelbach and David Carlisle. This package corrects a few problems
% in the LaTeX2e kernel, the most notable of which is that in current
% LaTeX2e releases, the ordering of single and double column floats is not
% guaranteed to be preserved. Thus, an unpatched LaTeX2e can allow a
% single column figure to be placed prior to an earlier double column
% figure. The latest version and documentation can be found at:
% http://www.ctan.org/tex-archive/macros/latex/base/



%\usepackage{stfloats}
% stfloats.sty was written by Sigitas Tolusis. This package gives LaTeX2e
% the ability to do double column floats at the bottom of the page as well
% as the top. (e.g., "\begin{figure*}[!b]" is not normally possible in
% LaTeX2e). It also provides a command:
%\fnbelowfloat
% to enable the placement of footnotes below bottom floats (the standard
% LaTeX2e kernel puts them above bottom floats). This is an invasive package
% which rewrites many portions of the LaTeX2e float routines. It may not work
% with other packages that modify the LaTeX2e float routines. The latest
% version and documentation can be obtained at:
% http://www.ctan.org/tex-archive/macros/latex/contrib/sttools/
% Documentation is contained in the stfloats.sty comments as well as in the
% presfull.pdf file. Do not use the stfloats baselinefloat ability as IEEE
% does not allow \baselineskip to stretch. Authors submitting work to the
% IEEE should note that IEEE rarely uses double column equations and
% that authors should try to avoid such use. Do not be tempted to use the
% cuted.sty or midfloat.sty packages (also by Sigitas Tolusis) as IEEE does
% not format its papers in such ways.





% *** PDF, URL AND HYPERLINK PACKAGES ***
%
%\usepackage{url}
% url.sty was written by Donald Arseneau. It provides better support for
% handling and breaking URLs. url.sty is already installed on most LaTeX
% systems. The latest version can be obtained at:
% http://www.ctan.org/tex-archive/macros/latex/contrib/misc/
% Read the url.sty source comments for usage information. Basically,
% \url{my_url_here}.





% *** Do not adjust lengths that control margins, column widths, etc. ***
% *** Do not use packages that alter fonts (such as pslatex).         ***
% There should be no need to do such things with IEEEtran.cls V1.6 and later.
% (Unless specifically asked to do so by the journal or conference you plan
% to submit to, of course. )


% correct bad hyphenation here
\hyphenation{op-tical net-works semi-conduc-tor}


\begin{document}
%
% paper title
% can use linebreaks \\ within to get better formatting as desired
\title{Bare Demo of IEEEtran.cls for Conferences}


% author names and affiliations
% use a multiple column layout for up to three different
% affiliations
\author{\IEEEauthorblockN{Michael Shell}
\IEEEauthorblockA{School of Electrical and\\Computer Engineering\\
Georgia Institute of Technology\\
Atlanta, Georgia 30332--0250\\
Email: http://www.michaelshell.org/contact.html}
\and
\IEEEauthorblockN{Homer Simpson}
\IEEEauthorblockA{Twentieth Century Fox\\
Springfield, USA\\
Email: homer@thesimpsons.com}
\and
\IEEEauthorblockN{James Kirk\\ and Montgomery Scott}
\IEEEauthorblockA{Starfleet Academy\\
San Francisco, California 96678-2391\\
Telephone: (800) 555--1212\\
Fax: (888) 555--1212}}

% conference papers do not typically use \thanks and this command
% is locked out in conference mode. If really needed, such as for
% the acknowledgment of grants, issue a \IEEEoverridecommandlockouts
% after \documentclass

% for over three affiliations, or if they all won't fit within the width
% of the page, use this alternative format:
% 
%\author{\IEEEauthorblockN{Michael Shell\IEEEauthorrefmark{1},
%Homer Simpson\IEEEauthorrefmark{2},
%James Kirk\IEEEauthorrefmark{3}, 
%Montgomery Scott\IEEEauthorrefmark{3} and
%Eldon Tyrell\IEEEauthorrefmark{4}}
%\IEEEauthorblockA{\IEEEauthorrefmark{1}School of Electrical and Computer Engineering\\
%Georgia Institute of Technology,
%Atlanta, Georgia 30332--0250\\ Email: see http://www.michaelshell.org/contact.html}
%\IEEEauthorblockA{\IEEEauthorrefmark{2}Twentieth Century Fox, Springfield, USA\\
%Email: homer@thesimpsons.com}
%\IEEEauthorblockA{\IEEEauthorrefmark{3}Starfleet Academy, San Francisco, California 96678-2391\\
%Telephone: (800) 555--1212, Fax: (888) 555--1212}
%\IEEEauthorblockA{\IEEEauthorrefmark{4}Tyrell Inc., 123 Replicant Street, Los Angeles, California 90210--4321}}




% use for special paper notices
%\IEEEspecialpapernotice{(Invited Paper)}




% make the title area
\maketitle


\begin{abstract}
%\boldmath
There is a growing interest for coalitional concepts in the smart grid, spanning multiple objectives such as power losses reduction, benefits maximization or market stability... In this paper, we study coalitions of prosumers (agents that produce but also consume depending on time) which aim at selling energy to the grid. We reckon that the grid is able to specify two requirements (stability and minimum production) that enable a coalition to enter the market. Based on probability distributions of units available power, coalitions which are able to achieve low standard deviations are considered as stable coalitions. As it seems intuitively understandable that maintaining stable coalitions would be less expensive in terms of communication load than highly versatible ones, our objective consists in forming coalitions that satisfy grid requirements both in terms of stability and minimum production. Our definition of stability being implicitely linked to correlation between agents timeseries (see section 2), we organize them over a correlation graph and use a variation of clique percolation to construct valid coalitions in reasonable time.
\end{abstract}
% IEEEtran.cls defaults to using nonbold math in the Abstract.
% This preserves the distinction between vectors and scalars. However,
% if the conference you are submitting to favors bold math in the abstract,
% then you can use LaTeX's standard command \boldmath at the very start
% of the abstract to achieve this. Many IEEE journals/conferences frown on
% math in the abstract anyway.

% no keywords




% For peer review papers, you can put extra information on the cover
% page as needed:
% \ifCLASSOPTIONpeerreview
% \begin{center} \bfseries EDICS Category: 3-BBND \end{center}
% \fi
%
% For peerreview papers, this IEEEtran command inserts a page break and
% creates the second title. It will be ignored for other modes.
\IEEEpeerreviewmaketitle



\section{Introduction}
% no \IEEEPARstart
One of the most basic ideas in the smart grid revolves around the introduction of communication means in the power grids, which could enable complex improvements in the energy management and lead progressively to a greener energetic system [1] [2]. Distributed energy resources (DER) such as wind turbines or photovoltaic panels are supposed to emerge and populate not only spread out farms, but also classical residential neighborhoods. Together with electric vehicles, and demand side management tools, they constitute the building blocks which will help turn the today pure energy consumers into true actors of the grid operation [3]. Such energy aware agents that consume, produce, and are ready to make concessions (appliances delays, V2G…) to ensure grid stability are commonly called prosumers [4] [5].

It is rather clear that there should be communication flows between prosumers and the grid as most of the demand side management concepts suppose such a link [6] [7]. Nevertheless as the expected number of active prosumers in the future seems quite large, the literature assume more and more communication between prosumers as this is a key ingredient to maintain more complex structures (such as coalitions), and thereby decrease the communication load on the grid’s side [8]. Furthermore, it has been shown that coalitions in the smart grid provide nice ways of improving stability while posing exciting challenges in terms of network infrastructure and management. Self sufficient microgrids that can disconnect from the main grid [9], virtual power plants (VPP) that map small DER in sizeable and adjustable plants [10], or coalitions of electric vehicles that back up the grid in emergency situations (V2G) [11] are just a few examples of how coalitional concepts can enhance grid reliability.

In this paper, our focus is on how coalitions of prosumers, aiming at selling energy to the grid, should be formed in order to support it efficiently. Namely, improving statistically production stability while decreasing communication needs for staying in stable state. The idea is that, by using prediction techniques, coalitions can propose contract values to the grid, enabling it to schedule production on its side accordingly. Actually, contract values are particularly relevant in power exchanges contexts where energy is traded based on predictions (day-ahead estimations for instance). It seems obvious that, in such systems, participants should try to minimize prediction errors in order to maintain a stable state of grid operation, and that some penalty rules should be applied to ensure that coalitions are willing to report contract values correctly. Furthermore, as the coalition productions are supposed to come from (uncertain) renewables, the notion of reliability comes into play with a visible tradeoff between contract values worth and reliability.

In this paper, our concern is on how prosumers should be aggregated in order to optimize this tradeoff, that is, how coalitions should be formed with the intention of announcing “high contract values with high reliability”. We thus aim at:

\begin{itemize}
\item Realistic representations (prosumers productions are non independent variables and come (indirectly) from real traces)
\item Minimizing the standard deviations of coalitions production probability distributions through the use of a proper utility function (see section 2)
\item Being (relatively) scalable with the number of prosumers
\end{itemize}

In more details, and as will be explained in section 2, we consider agents that, depending on meteorological conditions, personal preferences, and their appliances (loads and generators), consume and produce more or less energy (using only wind turbines and PV as generators, which can be quite simply and efficiently modeled).  We used real meteorological traces [13] to account for seasonal and daily variations in the prosumers energetic output profiles as well as realistic correlations. With these ingredients, we are able to run simulations and record the different output profiles as timeseries.

As correlations play a significant role in our mechanism, we organize the prosumers in a correlation graph, a quite popular approach in market stocks clustering [16] [17] (see section 3). We will see that in such graphs, cliques represent structures of interest. In our settings, it means that cliques are “good” (from the utility point of view) foundations for high valued coalitions. We thereby use a slightly modified clique percolation algorithm (see section 4) that will enable us to expand the cliques as much as needed in order to form proper coalitions that fulfill grid requirements (see results in section 5).



\section{Related Work}
Related work text here.


\section{Model}
Smart grid perspectives assume that smart meters populate the network, recording production/consumption values and reporting these to data centers. Any entities (DER, prosumers, microgrids, VPP and so on) would thereby be characterizable through timeseries expressing how much these entities produce or consume. 

Through our model, we wish thus to be able to generate, for any prosumer, its available production over time, which depend both on its pure production and consumption. Naturally, these are far from random in the sense that there are recurring patterns (consumption or expected solar irradience for instance) or statistical properties (see wind speed studies) that shape these profiles. We thus used meteorological traces as base inputs because they naturaly provide these key ingredients (sources). On the production side, as we only consider wind turbines and photovoltaic panels as generators, we concentrate primarily on wind speed and solar irradience traces. Variables such as temperature or humidity that correlates very well with consumption were also of great use. 

Basically, the user only has to indicate the number of prosumers and the simulation dates. Prosumers are then positioned randomly on a squared lattice. Any point of this lattice is linked to a meteorological profile derived from the input traces. A few parameters individualizing the prosumers (number and kind of DER owned, consumption habits, and so on...) are also picked randomly and account for diversity.

We denote by $ \mathcal{A} = \{ a_{1},...a_{N} \} $ the set of agents (the prosumers) and $ P_{i}(t) $ the instantaneous power value of agent i at instant t (its instantaneous production minus consumption, meaning that $ P_{i}(t) $ represents agent i available power at instant t). During the simulation (from $t_{0} $ to $ t_{K} $), all agents record their values with an hour time interval, we denote by $ \mathcal{T}_{i} = \{ P_{i}(t_{0}),...,P_{i}(t_{K}) \} $ the resulting timeserie of agent i. For any agent i, we note $ \mathcal{P}_{i} $ the probability distribution drawn from $ \mathcal{T}_{i} $ and by $ P_{i} $ a random variable following $ \mathcal{P}_{i} $.

We now extand these notations to any coalition $ S \subset \mathcal{A} $ : 
\begin{itemize}
\item $ P_{S}(t) = \sum_{i \in S} P_{i}(t) $
\item $ \mathcal{T}_{S} = \{ P_{S}(t_{0}),...,P_{S}(t_{K}) \} $ with $ \mathcal{P}_{S} $ its probability distribution
\item A coalition S has the possibility to announce a contract value $ P_{S}^{CRCT} $ on the market.
\end{itemize}

In this paper, we consider that coalition announcements are constrained by the grid. Namely, the grid has the possibility of fixing two parameters : 

\begin{itemize}
\item the reliability (denoted by $ \phi \in [0,1] $)
\item the minimum contract value under a given reliability ( $ P_{\phi}^{MIN} $). 
\end{itemize}

The reliability stipulates that for any coalition S willing to join the market, the probability of S value (at any instant t) being below its contract value should at most be $ \phi $. That is, $ \forall S,\ \forall t,\ Pr[P_{S}(t) \leq P_{S}^{CRCT} ] \leq \phi $. Furthermore, for consistency, we restrict $ \phi $ to small values ( $ \phi << 0.5 $). The minimum contract value under $ \phi $ states that only coalitions with higher contract values ($ P_{S}^{CRCT} \geq P_{\phi}^{MIN} $) will be accepted.

At this point, if a coalition S whishes to join the market, it has to choose a contract value that fulfill the two conditions. For simplification, we consider that coalitions will always apply the same economically consistent strategy of announcing the highest possible contract value that obeys the reliability rule (a value we denote by $ P_{\phi}(S) $). If $ P_{\phi}(S) \geq P_{\phi}^{MIN} $, meaning that it also obeys the minimum contract value rule, then S annonces this value on the market : $ P_{S}^{CRCT} = P_{\phi}(S) $. Otherwise, coalition S is not able to enter.
Basically, a coalition S is valid if and only if :
\[ \left\{ \begin{array}{lll}
		\forall t,\ Pr[ P_{S}(t) \leq P_{\phi}(S)] \leq \phi\ \textit{{\scriptsize (reliability rule)}} \\
		and\ P_{\phi}(S) \geq P_{\phi}^{MIN}\ \textit{{\scriptsize (min value rule)}}

\end{array} \right. \]

We now choose a very simple utility function that derives directly from the above remarks. If a coalition cannot provide a valid contract value, it receives naturally a utility of zero. Furthermore, it seems obvious that the utility should increase with the contract value. The $ 1/|S| $ term indicates that we favorise small coalitions, mainly because they are easier to maintain in term of communication.

\[ \mathcal{U}_{\phi,\ P_{\phi}^{MIN}}(S) = \mathbf{1}_{\textit{S\ valid}} \dfrac{P_{\phi}(S)}{|S|} \]

Obviously, maximising this utility function amounts to maximizing the coalition contract value with the minimum possible number of agents. 

In order to illustrate what is done in the following, lets consider a very simple example with two agents, say i and j, with gaussian value probability distributions $ \mathcal{P}_{i} = \mathcal{N}(\mu_{i}, \sigma_{i}) $ and $ \mathcal{P}_{j} = \mathcal{N}(\mu_{j}, \sigma_{j}) $, such that the joint probability distribution  $ \mathcal{P}_{ij} $ of the coalition $\{ij\}$ is also a gaussian with the following parameters :

\[ \left\{ \begin{array}{lll}
		\mu_{ij} = \mu_{i} + \mu_{j} \\
		\sigma_{ij} = \sqrt{\sigma_{i}^{2} + \sigma_{j}^{2} + \rho_{ij}\sigma_{i}\sigma_{j}}
\end{array} \right. \]

with $ \rho_{ij} $ the Pearson correlation coefficient between $ P_{i} $ and $ P_{j} $. We can easily write the reliability condition $ Pr[P_{ij}(t) \leq P_{\phi}(ij) ] \leq \phi $ as :

\[ \dfrac{1}{2} \left[ 1+ erf \left( \dfrac{P_{\phi}(ij) - \mu_{ij}}{\sigma_{ij}\sqrt{2}} \right) \right] \leq \phi \]

The strategy of $ \{ij\} $ consists in maximizing its contract values as long as it respects this inequality, e.g to annonce $ P_{\phi}(ij) = \mu_{ij} - \sigma_{ij}\sqrt{2}erf^{-1}(1-2 \phi ) $. It thus appears (as it was intuitively understandable) that, for equivalent sizes, coalitions with low relative standard deviations ( $ \sigma_{ij} / \mu_{ij} $ ) are able to announce higher contract values. 

What this paper investigates in the following is the developement of an algorithm that organizes prosumers such that the synergy term of the standard deviation ( $ \rho_{ij}\sigma_{i}\sigma_{j} $ in the example above) is minimized. In such settings, and through the clique percolation procedure (section 4), we will show that coalitions with low relative standard deviation, e.g high utility coalitions, can be computed.

\section{Correlation Graphs}

When dealing with numerous timeseries, a common objectiv consists in finding clusters whose elements exhibit high similarities while elements in different clusters are very dissimilar. This problem is usually approached by means of a similarity measure and completed by a clustering technique such as K-means or hierarchical clustering. In [] [], financial stocks are clustered according to their daily log return timeseries. Similarity measures based on Pearson correlation coefficient are established ($ d_{ij} = \dfrac{1}{2}\sqrt{2(1-\rho_{ij})}\ or\ d_{ij} = 1 - \rho_{ij}^{2} $ ) and a complete graph $ G = (V,E,\omega) $ is constructed such that a vertex stands for a stock and the weight of the edge connecting i and j is equal to $ d_{ij} $. In this complete state, the graph G is of little use, but several techniques for filtering information have been proposed such as minimum spanning tree [], k-nearest graphs [], or $\epsilon$-graphs [][].

The idea of $\epsilon$-graphs consists in filtering edges based on their weight, meaning only keeping edges such that their weight is less than $ \epsilon $. Despite the procedure simplicity, the choice of $ \epsilon $ (or k for k-nearest graphs) is not trivial and may influence strongly the results. There is indeed to our knowledge no optimal rule for chosing $ \epsilon $. In [], the authors studied the topological properties (mainly average clustering and connectivity) of the correlation graph depending on the threshold $ \epsilon $. One of the conclusions was that "strong links" (i.e links between strongly correlated timeseries) are responsible for clustering while "weak links" provide network connectivity (in such a network, weak links are indeed connecting densely connected communities).

Applying this technique to the production timeseries of our prosumers (after seasonal effects had been removed) leads to well defined clusters. Indeed, depending on where they are (work, home...), groups of people consume differently but similarly with others of the same group. In the same maner, groups of DER that are subject to similar meteorological conditions produce accordingly. Nevertheless, these clusters of strongly correlated prosumers are the exact opposite of what we are seeking. We thus opt for reversing the metric ($ d_{ij} = \rho_{ij}^{2} $) such that "strong links" become links between uncorrelated timeseries and "weak links", between (anti)correlated timeseries. As expected, independently of the $ \epsilon $ parameter, the graph exhibits henceforth much less clustering and communities are hardly visible. Therefore, using classical clustering or community detection algorithms do not provide very good results (little better than a random algorithm from the utility point of view). However, as it is intuitively understandable, cliques of this graph exhibit very good utility values. But, as some of them may be too small to reach the grid minimum value requirement or because some of them may benefit from additional agents even if they don't form a clique anymore, we used a slightly modified clique percolation algorithm in order to expand them while still controling their expansion. Next section provide some details on clique percolation. 

\section{Clique Percolation}

Communities in networks are often seen as groups of nodes exhibiting high internal densities of links as well as a low density across communities [1]. The clique percolation algorithm uses directly this observation and the idea behind it is actually quite simple [2] [3]. It starts indeed by searching for cliques (a complete subgraph) in the given graph and considers them as potential seeds for the different communities.

The next ingredient the algorithm needs in order to make the seeds grow is a fitness function that gives some value to a group of nodes. In community detection, it is of common use to employ topological functions that compare internal and external degree, but other functions can also be well suited depending on the problem. Seeds will then alternatively look in their respective neighborhood, select the node that yields the best increase in fitness, and finally incorporate it within the seed. The algorithm stops when seeds cease to grow or when all nodes are affected to at least one seed.
At this point, some coalitions may be partially overlapping, which can be of great use especially in social networks where persons often belongs to several communities (family, friends, work colleagues...). For simplicity, in this paper, we wish to keep the coalitions separated and  leave the management of overlapping coalitions for future work. We thus implemented a simple heuristic that consider nodes in multiple seeds one by one and chooses its final coalition as the one that “needs it the most” in terms of utility loss. More formally, for a coalition S and a node $ i \in S $, we define :

\[ \tau_{i}(S) = \dfrac{\mathcal{U}_{\phi,\ P_{\phi}^{MIN}}(S) - \mathcal{U}_{\phi,\ P_{\phi}^{MIN}}(S-\{i\}) }{\mathcal{U}_{\phi,\ P_{\phi}^{MIN}}(S)} \]

If node i belongs to  multiple coalitions, the only coalition retaining node i is the one that maximizes $ \tau_{i} $.

We now test our implementation of the clique percolation against well-known benchmark networks in community detection, namely the Zachary karate club, and the UK faculty network. Results can be seen in figure 1.

Left graph is the Zachary karate club graph, shapes represents real communities and color the algorithm output. We can see that only one node was misclassified. Right graph is the UK faculty graph where the algorithm is able to split the graph in three distinct communities.


\section{Results}

We now test the algorithm on a 200 prosumers example. Simulations were run from february 2006 to december 2010 such that we are dealing with 200 hourly sampled timeseries of available power over this period. The first step consists in removing seasonal effects attributable to wind, sun, and temperature natural patterns. We then construct the complete (un)correlation graph as explained in section 3. 

At this point, we have four degrees of freedom, namely the reliability ($ \phi $), the required power to enter the market ($P_{\phi}^{MIN}$), the prunning parameter ($\epsilon$) in order to obtain the $\epsilon$-graph, and the number of desired coalitions ($ N_{COAL} $). A first remark would be that $ \epsilon $ has a strong impact on the number of possible coalitions as it is directly responsible for the number and sizes of the cliques in the $ \epsilon$-graph. If $\epsilon $ is too low, the $\epsilon$-graph will not provide enough cliques, conversely, if $\epsilon $ is too high, we loose important information and cliques are able to percolate without topological constraints, leading to completely overlapping giant coalitions. Besides, high values on $\epsilon $ give rise to higher computational time as more complete graphs are considered. For $ N_{COAL}=n$, we thus choose $\epsilon $ as the smallest threshold such that the resulting graph contains at least n non overlaping cliques.

The first two parameters ($\phi$ and $P_{\phi}^{MIN}$) shape the utility function such that, if $ \phi $ is close to zero, the reliability requirement is very high and only small values of $ P_{\phi}^{MIN}$ could lead to valid coalitions (and positive utility values). Conversely, the higher $\phi$, the less constraints are imposed to the coalitions and valid ones can arise for a larger spectrum of $ P_{\phi}^{MIN}$ (see figure 1, left plot). In the following, we fix the reliability to a given empirical value and observe how the coalitions evolves for different values of $P_{\phi}^{MIN}$.

As is visible on the right plot of figure 1, when grid constraints are fixed and only the number of coalitions varies, we can see an increase in social welfare (sum of coalitions utilities) up to a maximum point before a decrease. The reason is that increasing the number of coalitions allow, with our algorithm, more coalitions to be stable and enter the market, but there is a point where nodes bringing stability are not sufficient in coalitions to make them pass the grid requirements, and some coalitions start to fail with zero utility. Moreover, reckon that increasing $ N_{COAL} $ means also increasing $ \epsilon $, leading to graphs where information is flooded, meaning that the algorithm performances decrease.
Naturally, it is also visible that small values of $P_{\phi}^{MIN}$ lead to higher utilities because coalitions are able to announce higher contract values.

Now that the behavior of the utility is more clear, we have a look at how the algorithm performs. As a comparison, we use a completely random algorithm that only ask for a number of coalitions and partition the agents in a random fashion. For consistency, we always average the results of this algorithm over 100 realization and the errorbars in the plots stands for the standard deviations of the results.

% An example of a floating figure using the graphicx package.
% Note that \label must occur AFTER (or within) \caption.
% For figures, \caption should occur after the \includegraphics.
% Note that IEEEtran v1.7 and later has special internal code that
% is designed to preserve the operation of \label within \caption
% even when the captionsoff option is in effect. However, because
% of issues like this, it may be the safest practice to put all your
% \label just after \caption rather than within \caption{}.
%
% Reminder: the "draftcls" or "draftclsnofoot", not "draft", class
% option should be used if it is desired that the figures are to be
% displayed while in draft mode.
%
%\begin{figure}[!t]
%\centering
%\includegraphics[width=2.5in]{myfigure}
% where an .eps filename suffix will be assumed under latex, 
% and a .pdf suffix will be assumed for pdflatex; or what has been declared
% via \DeclareGraphicsExtensions.
%\caption{Simulation Results}
%\label{fig_sim}
%\end{figure}

% Note that IEEE typically puts floats only at the top, even when this
% results in a large percentage of a column being occupied by floats.


% An example of a double column floating figure using two subfigures.
% (The subfig.sty package must be loaded for this to work.)
% The subfigure \label commands are set within each subfloat command, the
% \label for the overall figure must come after \caption.
% \hfil must be used as a separator to get equal spacing.
% The subfigure.sty package works much the same way, except \subfigure is
% used instead of \subfloat.
%
%\begin{figure*}[!t]
%\centerline{\subfloat[Case I]\includegraphics[width=2.5in]{subfigcase1}%
%\label{fig_first_case}}
%\hfil
%\subfloat[Case II]{\includegraphics[width=2.5in]{subfigcase2}%
%\label{fig_second_case}}}
%\caption{Simulation results}
%\label{fig_sim}
%\end{figure*}
%
% Note that often IEEE papers with subfigures do not employ subfigure
% captions (using the optional argument to \subfloat), but instead will
% reference/describe all of them (a), (b), etc., within the main caption.


% An example of a floating table. Note that, for IEEE style tables, the 
% \caption command should come BEFORE the table. Table text will default to
% \footnotesize as IEEE normally uses this smaller font for tables.
% The \label must come after \caption as always.
%
%\begin{table}[!t]
%% increase table row spacing, adjust to taste
%\renewcommand{\arraystretch}{1.3}
% if using array.sty, it might be a good idea to tweak the value of
% \extrarowheight as needed to properly center the text within the cells
%\caption{An Example of a Table}
%\label{table_example}
%\centering
%% Some packages, such as MDW tools, offer better commands for making tables
%% than the plain LaTeX2e tabular which is used here.
%\begin{tabular}{|c||c|}
%\hline
%One & Two\\
%\hline
%Three & Four\\
%\hline
%\end{tabular}
%\end{table}


% Note that IEEE does not put floats in the very first column - or typically
% anywhere on the first page for that matter. Also, in-text middle ("here")
% positioning is not used. Most IEEE journals/conferences use top floats
% exclusively. Note that, LaTeX2e, unlike IEEE journals/conferences, places
% footnotes above bottom floats. This can be corrected via the \fnbelowfloat
% command of the stfloats package.



\section{Conclusion}
The conclusion goes here.




% conference papers do not normally have an appendix


% use section* for acknowledgement
\section*{Acknowledgment}


The authors would like to thank...





% trigger a \newpage just before the given reference
% number - used to balance the columns on the last page
% adjust value as needed - may need to be readjusted if
% the document is modified later
%\IEEEtriggeratref{8}
% The "triggered" command can be changed if desired:
%\IEEEtriggercmd{\enlargethispage{-5in}}

% references section

% can use a bibliography generated by BibTeX as a .bbl file
% BibTeX documentation can be easily obtained at:
% http://www.ctan.org/tex-archive/biblio/bibtex/contrib/doc/
% The IEEEtran BibTeX style support page is at:
% http://www.michaelshell.org/tex/ieeetran/bibtex/
%\bibliographystyle{IEEEtran}
% argument is your BibTeX string definitions and bibliography database(s)
%\bibliography{IEEEabrv,../bib/paper}
%
% <OR> manually copy in the resultant .bbl file
% set second argument of \begin to the number of references
% (used to reserve space for the reference number labels box)
\begin{thebibliography}{1}

\bibitem{IEEEhowto:kopka}
H.~Kopka and P.~W. Daly, \emph{A Guide to \LaTeX}, 3rd~ed.\hskip 1em plus
  0.5em minus 0.4em\relax Harlow, England: Addison-Wesley, 1999.

\end{thebibliography}




% that's all folks
\end{document}


